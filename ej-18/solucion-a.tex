\noindent Con la situación y los valores iniciales anteriormente mencionados procedemos a hallar la distancia del punto $B$ al plano $\alpha$, para ello hallamos el módulo de la proyección $\vec{p}$:
\begin{center}
	$|\vec{p}|  =\cfrac{|\vec{b} \cdot \vec{n}|}{|\vec{n}|} \hspace*{1cm} \vec{b}=(2, 1, -1) \hspace*{1cm} \vec{n}=(3, 2, -5)$
\end{center}
\begin{multicols}{2}
	\noindent Hallamos el producto escalar entre $\vec{b} \cdot \vec{n}$:
	\begin{align*}
		\vec{b} \cdot \vec{n} & = (2) \cdot (3) + (1) \cdot (2) + (-1) \cdot (-5) \\
		\vec{b} \cdot \vec{n} & = 6 + 2 + 5                                       \\
		\vec{b} \cdot \vec{n} & = \boxed{13}
		\columnbreak
	\end{align*}
	\noindent Hallamos el módulo de $\vec{n}$:
	\begin{align*}
		|\vec{n}| & = \sqrt{(3)^2 + (2)^2 + (-5)^2} \\
		|\vec{n}| & = \sqrt{9 + 4 + 25}             \\
		|\vec{n}| & = \boxed{\sqrt{38}}
	\end{align*}
\end{multicols}
\noindent Hallamos el módulo de la proyección de $\vec{b}$ sobre $\vec{n}$:

$|\vec{p}| =\cfrac{13}{\sqrt{38}} \implies \cfrac{13}{38}\sqrt{38}$

\noindent $\therefore$ \ La distancia del punto $B$ al plano $\alpha$ es \ \fcolorbox{black}{yellow}{$\cfrac{13}{38}\sqrt{38}$}  $\approx 2,109$