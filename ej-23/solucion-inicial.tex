\subsubsection*{Valores iniciales:}

\begin{multicols}{2}
	\noindent En $r_1)$: \\
	\indent $\boxed{P_1(2, -1, 3)}$ \\
	\indent $\boxed{\vec{u}=(2, -1, 3)}$

	\columnbreak
	\noindent En $r_2)$: \\
	\indent $\boxed{P_2(1, 0, 2)}$

	\indent $\vec{v}  =\overrightarrow{PQ}     \\
		\indent \vec{v}  = (-2, 2, 6) - (1, 0, 2) \\
		\indent \vec{v}  =(-2-1, \ 2-0, \ 6-2)    \\
		\indent \boxed{\vec{v}  =(-3,2,4)}$
\end{multicols}

\noindent $\overrightarrow{P_1P_2} = (1, 0, 2) - (2, -1, 3)$

\noindent $\overrightarrow{P_1P_2} = [1 - 2, \ 0 - (-1), \ 2 - 3]$

\noindent $\boxed{\overrightarrow{P_1P_2} = (-1, 1, -1)}$
\vspace{1cm}

\noindent \textbf{Verificación de coplanaridad entre $r_1$ y $r_2$}:

\noindent Para verificar si las rectas son coplanares o alabeadas calculamos el producto mixto $\overrightarrow{P_1P_2} \cdot (\vec{u} \times \vec{v})$:
\begin{align*}
	\vec{u} \times \vec{v} & = (2, -1, 3) \times (-3, 2, 4)                                                          \\
	\vec{u} \times \vec{v} & = \begin{vmatrix}
		                           i  & j  & k \\
		                           2  & -1 & 3 \\
		                           -3 & 2  & 4 \\
	                           \end{vmatrix}                                                                        \\
	\vec{u} \times \vec{v} & = \vec{i} [(-1)(4) - (3)(2)] + \vec{j} [(3)(-3) - (2)(4)] + \vec{k} [(2)(2) - (-1)(-3)] \\
	\vec{u} \times \vec{v} & = \vec{i} (-4 - 6) + \vec{j} (-9 - 8) + \vec{k} (4 - 3)                                 \\
	\vec{u} \times \vec{v} & = \vec{i} (-10) + \vec{j} (-17) + \vec{k} (1)                                           \\
	\vec{u} \times \vec{v} & = (-10, \ -17, \ 1)
\end{align*}
\begin{align*}
	\overrightarrow{P_1P_2} \cdot (\vec{u} \times \vec{v}) & = (-1, 1, -1) \cdot (-10, -17, 1)      \\
	\overrightarrow{P_1P_2} \cdot (\vec{u} \times \vec{v}) & = [(-1)(-10)] + [(1)(-17)] + [(-1)(1)] \\
	\overrightarrow{P_1P_2} \cdot (\vec{u} \times \vec{v}) & = (10 - 17 -1)                         \\
	\overrightarrow{P_1P_2} \cdot (\vec{u} \times \vec{v}) & = (10 - 18)                            \\
	\overrightarrow{P_1P_2} \cdot (\vec{u} \times \vec{v}) & = \boxed{- 8}
\end{align*}

\noindent $\therefore$ \ podemos determinar que $r_1$ y $r_2$ son rectas \textbf{alabeadas}.






