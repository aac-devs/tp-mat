\noindent Planos paralelos a ambas rectas serán paralelos a $\omega$. Se podría usar el mismo vector normal $\vec{n}$ y hallar los puntos de paso a unidades del origen de coordenadas.

\noindent Valores iniciales:
\begin{center}
	$\vec{n} = (-13, -3, 4)$
\end{center}

\begin{multicols}{2}
	\noindent Hallamos el módulo de $\vec{n}$:
	\begin{align*}
		\abs{\vec{n}} & = \sqrt{(-13)^2 + (-3)^2 + (4)^2} \\
		\abs{\vec{n}} & = \sqrt{169 + 9 + 16}             \\
		\abs{\vec{n}} & = \sqrt{194}
	\end{align*}
	\columnbreak \\
	\noindent Hallamos el versor de $\vec{n}$:
	\begin{align*}
		\vec{n_0} & = \dfrac{1}{\abs{\vec{n}}} \cdot \vec{n}                                             \\
		\vec{n_0} & = \dfrac{1}{\sqrt{194}} \cdot (-13, -3, 4)                                             \\
		\vec{n_0} & = \left(\dfrac{-13}{\sqrt{194}}, \dfrac{-3}{\sqrt{194}}, \dfrac{4}{\sqrt{194}} \right)
	\end{align*}
\end{multicols}

\noindent Hallamos un vector $\vec{n_1} = 10 \cdot \vec{n_0}$ que nos servirá para hallar los puntos $P_1$ y $P_2$ que equidistan del plano en direcciones opuestas en 10 unidades:
\begin{align*}
	\vec{n_1} & = 10 \cdot \vec{n_0}                                                                           \\
	\vec{n_1} & = 10 \cdot \left(\dfrac{-13}{\sqrt{194}}, \dfrac{-3}{\sqrt{194}}, \dfrac{4}{\sqrt{194}} \right) \\
	\vec{n_1} & = \left(\dfrac{-130}{\sqrt{194}}, \dfrac{-30}{\sqrt{194}}, \dfrac{40}{\sqrt{194}} \right)       \\
\end{align*}
\begin{center}
	$\therefore \ \boxed{P_1 = \left(\dfrac{-130}{\sqrt{194}}, \dfrac{-30}{\sqrt{194}}, \dfrac{40}{\sqrt{194}} \right)}
		\ \land \ P_2 = -P_1 \implies
		\boxed{P_2 = \left(\dfrac{130}{\sqrt{194}}, \dfrac{30}{\sqrt{194}}, \dfrac{-40}{\sqrt{194}} \right)}$
\end{center}

\vspace{1cm}
\noindent A continuación hallamos las ecuaciones de los planos $\omega_1$ y $\omega_2$:

\noindent Usamos $P_1$ para hallar $\omega_1$:
\begin{align*}
	-13x - 3y + 4z + d                                 & = 0                     \\
	-13 \cdot \left( \dfrac{-130}{\sqrt{194}}\right)
	- 3 \cdot \left( \dfrac{-30}{\sqrt{194}}\right)
	+ 4 \cdot \left( \dfrac{40}{\sqrt{194}}\right) + d & = 0                     \\
	\dfrac{1690}{\sqrt{194}} + \dfrac{90}{\sqrt{194}}
	+ \dfrac{160}{\sqrt{194}} + d                      & = 0                     \\
	\dfrac{1940}{\sqrt{194}} + d                       & = 0                     \\
	10\sqrt{194} + d                                   & = 0                     \\
	d                                                  & = \boxed{-10\sqrt{194}}
\end{align*}
$\implies \ \omega_1) \ -13x - 3y + 4z -10\sqrt{194} = 0$

\noindent Usamos $P_2$ para hallar $\omega_2$:
\begin{align*}
	-13x - 3y + 4z + d                                  & = 0                    \\
	-13 \cdot \left( \dfrac{130}{\sqrt{194}}\right)
	- 3 \cdot \left( \dfrac{30}{\sqrt{194}}\right)
	+ 4 \cdot \left( \dfrac{-40}{\sqrt{194}}\right) + d & = 0                    \\
	- \dfrac{1690}{\sqrt{194}} - \dfrac{90}{\sqrt{194}}
	- \dfrac{160}{\sqrt{194}} + d                       & = 0                    \\
	- \dfrac{1940}{\sqrt{194}} + d                      & = 0                    \\
	- 10\sqrt{194} + d                                  & = 0                    \\
	d                                                   & = \boxed{10\sqrt{194}}
\end{align*}
$\implies \ \omega_2) \ -13x - 3y + 4z + 10\sqrt{194} = 0$

\vspace{1cm}
\noindent $\therefore$ Las ecuaciones de los planos paralelos a $r1$ y $r2$ que se encuentran a 10 unidades del origen de coordenadas son:

\begin{center}
	$\fcolorbox{black}{yellow}{$\omega_1) \ -13x - 3y + 4z -10\sqrt{194} = 0$} \hspace*{0.5cm}$ y
	$\hspace*{0.5cm} \fcolorbox{black}{yellow}{$\omega_2) \ -13x - 3y + 4z + 10\sqrt{194} = 0$}$
\end{center}

\vspace{2cm}
\noindent \textbf{Gráfica 12-c}:
\begin{center}
	\href{https://www.geogebra.org/3d/fmkxhwqc}{\includegraphics[width=15cm, scale=1]{TP-MATEMÁTICA-EJ12C.png}}
\end{center}
