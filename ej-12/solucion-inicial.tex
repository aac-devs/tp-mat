\subsubsection*{Valores iniciales:}

\begin{multicols}{2}
	$r1)
		\begin{cases}
			3x - 2y + z - 5 = 0 \hspace{1cm} (\alpha) \\
			2x + y - z - 5 = 0  \hfill (\beta)
		\end{cases}$

	$r2) \ \dfrac{x - 1}{2} = \dfrac{y - 3}{10} = \dfrac{x + 2}{14}$
\end{multicols}

\noindent \textbf{A continuación buscamos los vectores $\vec{u_1} \parallel r1$ y $\vec{u_2} \parallel r2$:}

\noindent Hallamos la recta que forma la intersección de los planos determinados en $r1)$:

\begin{center}
	$\vec{n_{\alpha}} = (3, -2, 1) \hspace*{1cm} \vec{n_{\beta}} = (2, 1, -1)$

\end{center}

\noindent Hallamos $\vec{u_1} \perp \vec{n_{\alpha}} \land \vec{n_{\beta}}$:
\begin{align*}
	\vec{u_1}              & = \vec{n_{\alpha}} \times \vec{n_{\beta}}          \\
	\vec{u_1}              & = \begin{vmatrix}
		                           i & j  & k  \\
		                           3 & -2 & 1  \\
		                           2 & 1  & -1
	                           \end{vmatrix}                                   \\
	\vec{u_1}              & = \vec{i}(2 - 1) - \vec{j}(-3 -2) + \vec{k}(3 + 4) \\
	\vec{u_1}              & = \vec{i}(1) - \vec{j}(-5) + \vec{k}(7)            \\
	\vec{u_1}              & = \vec{i} + 5\vec{j} + 7\vec{k}                    \\
	\therefore \ \vec{u_1} & = \boxed{(1, 5, 7)}
\end{align*}

\noindent Hallamos $\vec{u_2}$ de $r2$ igualando cada expresión a un $\lambda$ dado, con lo que obtenemos la siguiente ecuación paramétrica:

\begin{center}
	$r2)
		\begin{cases}
			x = 1 + 2\lambda  \\
			y = 3 + 10\lambda \\
			z = -2 + 14\lambda
		\end{cases}$ \\
	\vspace{0.3cm}
	\noindent $\therefore \ \boxed{\vec{u_2} = (2, 10, 14)}$  \\
	\vspace{0.3cm}
	Obtenemos también el punto $P_0(1,3,-2)$
\end{center}

